\documentclass[11pt]{article}
\usepackage{amsmath,amssymb}
\usepackage{tikz}
\usepackage[utf8]{inputenc}
\usepackage{circuitikz}
\usepackage{algorithm}
\usepackage{algpseudocode}

\title{Nash Stream Cipher: A Hardware-Optimized Implementation}
\author{[Author]}
\date{\today}

\begin{document}
\maketitle

\begin{abstract}
This document presents a modern hardware implementation of John Nash's stream cipher, originally proposed to the NSA in 1955. The implementation features auto-synchronization capabilities, error recovery, and resistance to side-channel attacks.
\end{abstract}

\section{Algorithm Specification}
\subsection{State Machine}
The cipher consists of two permutation paths (red and blue) through a state machine with the following properties:
\begin{itemize}
    \item State transitions defined by permutation functions
    \item Bit inversion operations (+/-) at specific states
    \item Auto-synchronization through feedback mechanism
\end{itemize}

\section{Security Analysis}
\subsection{Computational Security}
Nash's exponential conjecture states that for sufficiently complex enciphering systems, the computational work required to break them grows exponentially with key length...

\section{Hardware Implementation}
\subsection{Resource Requirements}
\begin{itemize}
    \item State machine logic
    \item Memory elements
    \item Permutation path routing
\end{itemize}

\end{document}